\documentclass[12pt]{spieman}  % 12pt font required by SPIE;
%\documentclass[a4paper,12pt]{spieman}  % use this instead for A4 paper
\usepackage{amsmath,amsfonts,amssymb}
\usepackage{graphicx}
\usepackage{setspace}
\usepackage{tocloft}

\usepackage{ulem}
\usepackage{marginnote}
\newcommand{\jrmadd}[1]{\textcolor{red}{#1}}
\newcommand{\jrmrmv}[1]{\textcolor{red}{\sout{#1}}}
\newcommand{\jrmcom}[1]{\textcolor{red}{[#1]}} 

% \title{Three Sided Pyramid Wavefront Sensor Experiment on the CACTI Testbed}

\title{The Three Sided Pyramid Wavefront Sensor. II.  Experimental Demonstration on the CACTI Testbed}
\author[a,*]{Lauren Schatz}
\author[b]{Katniss Everdeen the Kitty on Fire}
\author[c]{Alexander the Great Long Cat}
\author[d]{Star the Hedgie}
%\author[b]{Third Author}
%\author[a,b,*]{Fourth Author}
\affil[a]{University Name, Faculty Group, Department, Street Address, City, Country, Postal Code}
\affil[b]{Cozy pile of blankets}
\affil[c]{The very top of the cat tree}
\affil[d]{Deep inside the snuggle sack}

\renewcommand{\cftdotsep}{\cftnodots}
\cftpagenumbersoff{figure}
\cftpagenumbersoff{table} 
\begin{document} 
\maketitle

\begin{abstract}
Abstract here. 
\end{abstract}

% Include a list of up to six keywords after the abstract
\keywords{cats, photonics, light, lasers, journal manuscripts, LaTeX template}

% Include email contact information for corresponding author
{\noindent \footnotesize\textbf{*}Lauren Schatz,  \linkable{laurenhschatz@gmail.com} }

\begin{spacing}{2}   % use double spacing for rest of manuscript

\section{Introduction}
\label{sect:intro}  % \label{} allows reference to this section



High contrast imaging is a technique used to image faint companions next to bright stars. Targets include circumstellar disks \cite{rodigas2014morphology}, active galactic nuclei \cite{imanishi2020subaru}, and exoplanets\cite{bowler2016imaging}. Known exoplanets have contrasts of 10$^-4$ to 10$^-10$ to the host stars. These high contrast ratios present challenges in directly imaging faint objects. In direct imaging the object is resolved spatially from the star, allowing for an image of the faint companion to be taken. Overcoming this contrast problem requires a two-fold solution. The starlight needs to be suppressed by a coronagraph, and the resulting high contrast region called the dark hole needs to be maintained over the course of the observation through extreme adaptive optics (ExA0) and wavefront sensing and control (WS$\&$C) techniques. ExAO systems operate by propagating light from a guide star to a wavefront sensor, which measures the phase error of the starlight wavefront. A computer then sends commands to shape a deformable mirror (DM) to correct for the phase error, forming a closed feedback loop that compensates for most of the atmospheric distortion. The corrected beam is then passed to a coronagraph which blocks the light from the on-axis star and allows us to detect faint off-axis companions. There are many types of coronagraphs currently used in high contrast imaging; most work by blocking out light using masks and stops\cite{soummer2004apodized}, or using interferometric techniques to destructively interfere light at the focal plane.\cite{foo2005optical} Uncorrected phase errors and  non-common path errors from the ExAO system result in speckles in the focal plane that reduce coronagraph contrast.
% These speckles evolve with time\cite{goebel2016evolutionary} so further methods of WF$\&$C in addition to the ExAO closed loop is needed to null speckles \jrmadd{$\leftarrow$ correct, but relevant to this paper?}.  

% GET RID OF DESCRIPTIONS START LISTING, CURRENT TESTBEDS INCLUDE...
%  \jrmcom{How is coronagraphy relevant?  This ought to be in intro of dissertation overall, but not in this paper} \jrmrmv{One area of research is the development of different coronagraphs for telescopes that have complex apertures. There are also test-beds focused on advancing ExAO and WF$\&$C. This included techniques such as Linear Dark Field Control (LDF), Electric Field Conjugation (EFC), and Low Order Wavefront Sensing (LOWFS).}
The next generation of giant ground and space telescopes will have the light-collecting power to detect and characterize potentially habitable terrestrial exoplanets for the first time. The ground based Giant Segmented Mirror Telescopes (GSMTs) includes the Thirty Meter Telescope (TMT) \cite{chisholm2020thirty}, the Giant Magellan Telescope (GMT) \cite{fanson2020overview}, and the European Extremely Large Telescope (E-ELT)\cite{ramsay2020eso}. Proposed giant space telescope missions are the LUVIOR\cite{2020AAS...23544704O} and HabEx\cite{gaudi2020habitable} missions. Various testbeds are advancing technology and techniques to enable exoplanet imaging on the GSMTs. Current testbeds include the High Contrast Imager for Complex Aperture Telescopes (HiCAT)\cite{2014SPIE.9143E..27N} at the Space Telescope Science Institute,  High-Contrast High-Resolution Spectroscopy for Segmented Telescopes (HCST)\cite{jovanovic2018high} at the California Institute of Technology, the Decadal Survey Testbed\cite{belikov2014ames} at the NASA Jet Propulsion Laboratory (JPL), LAM-ONERA On-sky Pyramid Sensor (LOOPs) (CITE) at the Laboratoire d'Astrophysique de Marseille, the Occulting Mask Coronagraph\cite{shi2017dynamic} also at JPL, and the High Contrast High- Resolution Spectroscopy for Segmented telescopes testbed (HCST) (CITE) at the California Institute of Technology. The Magellan Extreme Adaptive Optics system (MagAO-X)\cite{males2020magao} developed for the Magellan Clay Telescope, doubles as an ExAO testbed. Predictive real time control algorithms are being explored on MagAO-X to improve ExAO correction.\cite{haffert2021data} MagAO-X employs a Vector Apodizing Phase Plate Coronagraph (vAPP)\cite{snik2012vector} to enable low order modal wavefront sensing.\cite{2019JATIS...5d9004M} The the Subaru Coronagraphic Extreme Adaptive Optics instrument (SCExAO)\cite{jovanovic2015subaru}, similarly is used to advance ExAO technology. There are many more high-contrast imaging testbeds in existence, and many are listed on the Community of Adaptive OpTics and hIgh Contrast testbeds website.(CITE) A recent summary of current coronagraphy testbeds for space missions can be found in the decadal white paper.\cite{mazoyer2019high}

The Comprehensive Adaptive Optics and Coronagraph Test Instrument (CACTI) was designed with the flexibility to support visiting instruments and to be easily re-configurable to perform multiple experiments. Here we describe the design of  CACTI and its current status. We then discuss an experiment performed on CACTI with a visiting three-sided pyramid wavefront sensor (3PWFS) to explore an alternative wavefront sensor architecture for GSMT-ExAO. The pyramid wavefront sensor (PWFS) acts like a Foucault test in two dimensions. Light from the telescope is focused onto a glass pyramid tip where it is split and then the pupil plane is re-imaged onto a detector. The result is copies of the telescope pupil that contain intensity fluctuations that are related to the wavefront phase. All current PWFS on telescopes use a four sided pyramid (4PWFS), resulting in four pupil images. The sampling of each of the pupils sets the number of modes the ExAO system can correct. ExAO systems require high sampling of wavefronts to optimize performance and as a result require large detectors. Detector size, speed, and noise all impose limits on wavefront sensor performance. The 3PWFS is less sensitive to read noise than the 4PWFS because it uses fewer detector pixels. The 3PWFS has further benefits; a high quality three-sided pyramid optic is easier to manufacture than a four-sided pyramid. Schatz et al. (CITE) determined in simulation that the 3PWFS meets and can exceed the performance of a 4PWFS for a detector with high read noise.  Both a 3PWFS and 4PWFS were integrated into  CACTI to demonstrate the operation of a 3PWFS and compare it to the 4PWFS. 

\section{Design of CACTI}
CACTI was designed to model a full end-to-end adaptive optics system and have the flexibility to support multiple experiments. In the configuration described here CACTI consisted of two components, an adaptive optics simulator and a pyramid wavefront sensor testbed. In the following sections we describe the optical design in detail. 



\subsection{Adaptive Optics Simulator}

Adaptive optics (AO) systems operate under the concept of conjugate imaging. Light from the guide star travels through the layers of Earth's atmosphere and acquires phase errors from atmospheric turbulence. In single conjugate AO we model the effects of atmospheric turbulence as a single phase screen at a known altitude that imparts all of the phase errors to the guide star's wavefront. This layer of turbulence is typically modeled to be in the entrance pupil and is imaged to conjugate planes in the AO system. The optical design of an AO system consists primarily of pupil relays that re-size and re-image the pupil plane onto the wavefront sensor and DM so that the turbulence screen is both sensed and corrected in the same plane. Adaptive optics systems work over a wavelength bandpass; to mitigate the effect of chromatic aberrations through the system mirrors are used as the optics in these pupil relays. In a closed loop AO system the wavefront sensor is placed after the DM. When the loop is closed the wavefront sensor sees a wavefront that is the combination of the atmospheric turbulence and the correction applied to the DM. There is a time lag between when the wavefront is sensed and when the correction applied which results in uncorrected wavefront errors that are quantified by the temporal error in the AO error budget. ExAO systems are designed to be low latency and run at loop speeds greater than 1-kHz.

\begin{figure}
    \centering
    \includegraphics[width=0.8\textwidth]{CACTI.png}
    \caption{In the current configuration CACTI consists of an Adaptive Optics Simulator and the PWFS testbed. Beamsplitters are used in CACTI to provide access to focal planes, collimated spaces, and additional sources. In the current configuration light is relayed through the AO simulator onto the 1024 actuator Boston Micromachine 1K (BMC1K) DM OAP mirrors. The light is then passed to the PWFS testbed which includes a modulation mirror, a 4PWFS and a 3PWFS.}
    \label{fig:CACTI}
\end{figure}

CACTI was designed to simulate atmospheric turbulence and a complete closed loop AO system. The optical design of  CACTI  is shown in Figure \ref{fig:CACTIZemax} and Table \ref{tab:CACTItable} summarizes the main components. The layout on the optical table is shown in Figure \ref{fig:CACTI}. A Helium-Neon (HeNe) laser (wavelength 633-$nm$) is used for inital alignment and testing.  The light is passed to a spatial filter with a 10-$\mu m$ pinhole to clean up any wavefront errors, and insure that the start of the system is an unresolved point source. The point source is then collimated by an off-axis parabolic (OAP) mirror to simulate starlight coming from infinity. There are six OAP mirrors in total that form the pupil relays of the system. All are cored from the same parent with $\lambda /10$ surface quality (Peak-to-Valley). Each OAP has a focal length of 375.25-$mm$ and an off-axis angle of 23 degrees. A 50/50 beamsplitter cube is placed into the collimated beam after the first OAP as an optional input for another collimated light source. After the beamsplitter a 7.5-$mm$ in diameter circular clear aperture mask is placed to define the entrance pupil of CACTI. The first pupil relay formed by the 2$^{nd}$ and 3$^{rd}$ OAP mirrors re-images the entrance pupil onto a flat mirror that is mounted on a kinematic base. The flat mirror is intended to be removed and replaced by a DM in the future. The second pupil relay created by the 4$^{th}$ and 5$^{th}$ OAP mirrors, relays the pupil onto a 1024 actuator Boston Micromachine 1K (BMC1K) DM. In the experiments described here, the BMC1K is used to simulate the atmosphere, correct the errors in closed-loop with the wavefront sensor, and correct for common path errors from misalignments. The last OAP focuses the light to the final focal plane in the AO simulator. A 50/50 beamsplitter is placed in this converging beam so that an additional focal plane can be accessed. In the current configuration of CACTI we have placed a Basler ACE CMOS camera as our science camera (Camsci) at this focal plane.

\begin{table}
	\begin{center}
		\begin{tabular}{ | l| l | }
			\hline
			\textbf{Component}& \textbf{Description}\\ \hline
			Light Source & Helium-Neon laser 633-$nm$ $\lambda$\\ \hline
			Spatial filter & 10-$\mu m$ Pinhole \\ \hline
			Off-axis parabolic mirrors & 375.25-mm Focal length, 23$^{\circ}$ off-axis angle \\ \hline
            Entrance pupil & 7.5-mm Clear aperture mask \\ \hline
            Boston Micromachines deformable mirror & 1024 Actuators, 9mm in diameter \\ \hline
            Beamsplitters & 50/50 Cube beamsplitter \\ \hline
            Modulation Mirror &  PI S-331 Piezo-actuator stage \\ \hline
            3PWFS Optic & Fused silica glass monolith \\ \hline
            4PWFS Optic & Crossed roof prisms \\ \hline
            Science Camera (Camsci) & Basler ACE acA640-750um CMOS\\ \hline
            3PWFS Camera (Camzyla) & Zyla 4.2+ sCMOS detector \\ \hline
            4PWFS Camera (Cam4p) & Basler ace acA720-520um CMOS \\ \hline
            Lens 1 (L1) & Doublet 500-mm focal length  \\ \hline
            Lens 2 (L2) & Doublet (CHECK ZEMAX)  \\ \hline
            Lens 3 (L3) & Custom achromatic air-spaced triplet  \\ \hline
            3PWFS Camera lens 1 (C1) & Doublet 30-mm focal length \\ \hline
            3PWFS Camera lens 2 (C2) & Doublet 30-mm focal length \\ \hline
            4PWFS Camera lens 1 (C3) & Doublet 50-mm focal length \\ \hline
            4PWFS Camera lens 2 (C4) & Doublet 30-mm focal length \\ \hline
				
			\end{tabular}
		\end{center}
	\caption{Descriptions of the components in CACTI.}
	\label{tab:CACTItable}
\end{table}

\begin{figure}
    \centering
    \includegraphics[width=0.8\textwidth]{CACTIzemax.png}
    \caption{Optical design of CACTI. Light starts from a point source from the HeNe laser and relayed through a series of afocal pupil relays. The focal plane of the AO simulator is after the 6${^th}$ OAP. It is collimated by a lens and passed into the PWFS testbed. A beamsplitter sends the same focal plane to both the 3PWFS and 4PWFS to minimize non-common path errors.}
    \label{fig:CACTIZemax}
\end{figure}

\subsection{Software and Calibration}

CACTI uses the Compute And Control For Adaptive Optics (CACAO) real-time control software package \cite{guyon2018compute}. The calibration of the AO system by CACAO is a multi-step process. Hadamard modes are applied to the DM, and the WFS response is recorded. The WFS signals from the Hadamard modes are then decomposed into the response of the influence functions from single actuators. These influence functions are then projected onto Fourier modes to create the basis set for the modal wavefront control. In these steps the illumination pattern of the deformable mirror is determined by thresholding actuators that don't give a response in the wavefront sensor. Similarly a mask of the PWFS detector pixels is generated to keep only the valid pixels from the PWFS pupils on the detector that will be used for wavefront sensing. CACAO is also used to generate the phase screens to simulate turbulence. Due to the limited low-order stroke of the BMC1K, the power-spectrum of the turbulence generated by CACAO is filtered to suppress the low order modes so that the full stroke of the DM is not used. At middle to high spatial frequencies the power spectrum matches that of Kolmogorov turbulence. 


\subsection{Pyramid Wavefront Sensor Testbed}

% The pyramid wavefront sensor testbed is a visiting experiment to compare the performance between a 3PWFS and a 4PWFS in varying level of turbulence. \jrmcom{don't you already say all this in the intro? check you organization of this and the intro:} The PWFS operates as a Foucault knife edge test in two dimensions. Light is focused onto the pyramid tip, and the edges of the pyramid divide the focal plane into segments. The apex angle of the pyramid imparts a tilt to each of the segments resulting in spatially separated copies of the pupil on the detector. The diffraction off of the pyramid edges encodes the phase error of the wavefront into the intensity pattern detected in the pupils via a Hilbert transform. CITE (verinaud, myself ;D). 
The PWFS has four main components, a focusing optic, the pyramid optic, a camera lens, and the detector. The pyramid optic needs to be high quality so that performance is not lost due to manufacturing errors. Common manufacturing errors of pyramid optics include chipping and roofing of the pyramid tip. To mitigate loss of performance a slow beam is focused onto the pyramid tip, so that the spot size is large compared to any potential defects. The size and spacing of the pupil on the detector is set by a combination of the apex angle of the pyramid and the camera lens. The spatial sampling of each of the pupils controls the number of modes the AO system can correct for, and the number of pixels sampling each pupil should be at least equal the number of actuators on the deformable mirror to optimize the AO system. There is a benefit to oversampling the pupils to avoid effects of aliasing at the cost of introducing more read noise. The PWFS is highly sensitive but suffers from a lack of dynamic range. To increase the linearity of the PWFS the PSF on the focal plane of the pyramid tip is modulated in a circle by a piezo-actuated mirror. The effect of modulation is to increase the linearity of the PWFS at the expense of sensitivity. 

ExAO systems need high sampling of the wavefront to optimize performance and as a result require larger detectors. An ideal wavefront sensing camera for ExAO has a large number of pixels that can be read out at a fast speed with low read noise. Current detectors are limited so there is a trade-off between detector, size, speed and performance. We are interested in exploring the 3PWFS as an alternative wavefront sensor for the GSMTs because the 3PWFS uses fewer detector pixels than the 4PWFS. Previous work has shown in simulation that because of this the 3PWFS is less sensitive to read noise, resulting in a modest boost in performance. (CITE myself). On CACTI we want to test the performance of a real 3PWFS, and expand the previous study to understand how the 3PWFS performs under different turbulence conditions compared to the 4PWFS. 

The design of the PWFS testbed is detailed in Figure \ref{fig:CACTI}. Light from the AO simulator is collimated by a 500-mm focal length lens (L1) which relays the exit pupil of the AO simulator to the PWFS testbed. The exit pupil of the AO simulator becomes the entrance pupil to the PWFS testbed, which is then resized by a pupil relay consisting of an achromatic doublet (L2) and a custom air-spaced achromatic triplet lens (L3). The pupil is imaged on the the modulation mirror, which is a $\lambda/20$ flat mirror mounted on a PI S-331 high speed piezo-actuator tip/tilt platform. A 3PWFS designed by Hartsci LLC has been integrated into the instrument for the performance test. The three sided pyramid optic is single prism made from fused silica glass and has a tip smaller than 5 microns that was custom made for this experiment. Figure \ref{fig:pyramidOptics}.A shows the 3D model of the manufactured prism. The 4PWFS in CACTI uses two crossed roof prisms for its pyramid optic shown in Figure \ref{fig:pyramidOptics}.B. This drawing is Figure 2. from Jovanovic et al. (2016cite{jovanovic2016scexao}). This is the same type of pyramid used in the Subaru Coronagraphic Extreme Adaptive Optics instrument (SCExAO)\cite{jovanovic2015subaru}.  A 50/50 beamsplitter after the modulation mirror sends the same PSF to the pyramid tips of the 3PWFS and 4PWFS to mitigate non-common path errors, as all the optics up to that point are common to both wavefront sensors. The PSF on the pyramid tips were sharpened by the DM using an eye-doctor script. The script is a grid search that applies different amplitudes of Zernike modes on the deformable mirror, and saves the combination of modes that maximizes the Strehl Ratio at the focal plane as the  DM set point where non-common path errors are minimized.

\begin{figure}
    \centering
    \includegraphics[width=0.8\textwidth]{pyramidOptics.png}
    \caption{Drawings of the pyramid optics in CACTI. A. The 3PWFS pyramid is a single prism made from fused silica glass. The 4PWFS pyramid is two crossed roof prisms. This pyramid is a copy of of the pyramid used by SCeXAO. }
    \label{fig:pyramidOptics}
\end{figure}


The pupil of each PWFS are imaged on the the detectors by using two camera lenses, C1 and C2 for the 3PWFS, and C3 and C4 for the 4PWFS,  that form a zoom lens system to insure that the sizes of pupils from both PWFS are 30 pixels in diameter. The 3PWFS uses a Zyla 4.2+ sCMOS detector which has an average read noise of (GO BACK AND FIND). The 4PWFS has a Basler ace acA720-520um CMOS camera that has an average read noise of (GO BACK AND FIND). Schatz et al. found in simulation that when the PWFS are well illuminated by light the effects of read noise to performance are negligible. The experiments performed in this paper were under bright light conditions, so having different cameras with different noise characteristics should not effect the PWFS performance. The PWFS signals on CACTI were processed using the the Raw Intensity (RI) method. The detector signal from the PWFS is dark subtracted, and a threshold is applied to mask out any pixels outside of the PWFS pupils. In the RI method the remaining signal is used as-is and are unraveled into a vector of intensity values that is used for calibration and wavefront reconstruction.


% %%%%%---------In case I go back and add in SM--------------
% The PWFS signals on CACTI can be processed in two ways, the Raw Intensity (RI) and Slopes Maps (SM) methods. In both methods the detector signal from the PWFS is dark subtracted, and a threshold is applied to mask out any pixels outside of the PWFS pupils. In the RI method the remaining signal is used as-is. The Slopes Maps calculation recombines the PWFS pupils into an estimate of the X and Y slopes of the wavefront slope. The SM equation for the 4PWFS is given in Equation \ref{4PWFSslopes}, and the Equation for the 3PWFS is given in \ref{3PWFSslopes}. In these equations $S_x, S_y$ are the local wavefront slopes, and $I_1...I_4$ are the intensity values of the pixel corresponding to the same location in each pupil.


% \begin{eqnarray}
%     S_x=\frac{I_1+I_2-I_3-I_4}{I_1+I_2+I_3+I_4}     \label{4PWFSslopes} \\
%     S_y=\frac{I_1-I_2-I_3+I_4}{I_1+I_2+I_3+I_4} \nonumber
% \end{eqnarray}

% \begin{eqnarray}
%     S_x=\frac{\frac{\sqrt{3}}{2}I_2-\frac{\sqrt{3}}{2}I_3}{I_1+I_2+I_3} \label{3PWFSslopes} \\
%     S_y=\frac{I_1-\frac{1}{2}I_2-\frac{1}{2}I_3}{I_1+I_2+I_3} \nonumber
% \end{eqnarray}

\subsection{Current Status}

The alignment of the adaptive optics simulator and PWFS tesbed in CACTI was completed in May 2020. Figure \ref{fig:cactiTestbed} is a picture of the as built system. Light from the HeNe laser is relayed by the six OAP mirrors, and is then coupled to the PWFS testbed using a beam triangle formed by two flat mirrors. Figure \ref{fig:PWFStestbed} is a close up diagram of the PWFS testbed. The first two lenses (L1 and L2) relay the entrance pupil of the PWFS testbed onto the PI modulation mirror (PI). A beamsplitter after the modulation mirror picks off light to the four-sided pyramid (4P) and the through beam is sent to the three-sided pyramid (3P). Both pyramids have two camera lenses (C1 and C2) that form a zoom lens to match the diameters of the pupils of the two PWFS. The measured pupil diameter for the 4PWFS is 30.5 pixels and 29.5 pixels for the 3PWFS. 

\begin{figure}
    \centering
    \includegraphics[width=0.8\textwidth]{cactiTestbed.png}
    \caption{Image of the as-built CACTI testbed. Light starts from the HeNe laser and propogates through the AO simulator on the left half of the optical table. After the BMC1K and the final OAP, the light is relayed to the PWFS testbed on the right half of the optical table.}
    \label{fig:cactiTestbed}
\end{figure}

\begin{figure}
    \centering
    \includegraphics[width=0.8\textwidth]{PWFStestbed.png}
    \caption{Close up image of the PWFS testbed. Light enters the system before L1 on the left side. The light is relayed onto the modulation mirror (PI). The through beam propagates through to the 3PWFS which consists of the pyramid optic (3P), camera lenses (C1$\&$2) and the Zyla CMOS camera (Camzyla). Light is by a beamsplitter (BS) to the 4PWFS arm which consists of the pyramid (4P), camera lenses (C3$\&$4) and the Basler CMOS camera (Cam4p).  }
    \label{fig:PWFStestbed}
\end{figure}


The PSF at the pyramid tip was optimized by an eye-doctor script to generate the flat commands for the deformable mirror. The system responses to the flat wavefront generated by the deformable mirror is given by Figure \ref{fig:flatCACTI}. Figure \ref{fig:flatCACTI}.A is the PSF in logarithmic scale on our science camera. Figure \ref{fig:flatCACTI}.B similarly is the PSF on the pyramid tip also in log scale. Figure  \ref{fig:flatCACTI}.C and Figure \ref{fig:flatCACTI}.D are the pyramid pupils from a flat wavefront with 5$\lambda/D$ modulation for the 3PWFS and 4PWFS respectively. There is a ghost in the CACTI PWFS located at the 4$^{th}$ Airy ring and caused by one of the lenses in the system. 

\begin{figure}
    \centering
    \includegraphics[width=0.8\textwidth]{flatCACTI.png}
    \caption{Signals from CACTI in response to the optimized flat wavefront. A. The PSF on the science camera. B. The optimized PSF on the focal plane of the PWFS tips. C. The 3PWFS pupils on Camzyla at 5 $\lambda/D$ modulation. D. The 4PWFS pupils on Cam4p at 5 $\lambda/D$ modulation.}
    \label{fig:flatCACTI}
\end{figure}


We have successfully closed the AO loop on CACTI with both the 4PWFS and 3PWFS using both the RI and SM signal handling methods. Closing the loop on this 3PWFS is a major milestone for the development of the three-sided pyramid wavefront sensor, because this is the first time the loop has been closed on a 3PWFS built with a real glass pyramid prism. Previous work by Schatz et al. closed the AO loop on a 3PWFS on the LOOPS testbed at the Laboratoire d'Astrophysique de Marseille. The 3PWFS in LOOPS was created by a phase screen applied to a spatial light modulator. The deformable mirror on CACTI was used to both generate the turbulence screen and apply the correction. This is done using the CACAO software which creates multiple channels of commands for the deformable mirror. In one channel we can stream the turbulence phase screens, and in another channel we have the commands computed by real time control software using the PWFS signals. The actual command applied to the DM is a summation of commands from all of the channels. Figure \ref{fig:turbCACTI} shows the closed loop PSFs and pyramid pupils from a turbulence strength of 0.3-$\mu m$ RMS error on CACTI. Figure \ref{fig:turbCACTI}.A and \ref{fig:turbCACTI}.B are the 3PWFS pupils and the closed loop PSF in log scale on the science camera. Figure \ref{fig:turbCACTI}.C and \ref{fig:turbCACTI}.D are the 4PWFS pupils and the closed loop PSF in log scale on the science camera.


\begin{figure}
    \centering
    \includegraphics[width=0.8\textwidth]{turbCACTI.png}
    \caption{A. Time-averaged science camera image of the 3PWFS closed-loop PSF in log scale. B. Turbulence streaming across the 3PWFS pupils. C. Time-averaged science camera image of the 3PWFS closed-loop PSF in log scale. D. Turbulence streaming across the 4PWFS pupils.}
    \label{fig:turbCACTI}
\end{figure}

% \begin{table}
% 	\begin{center}
% 		\begin{tabular}{ | l|l|l | }
% 			\hline
% 			\textbf{Camera}& \textbf{Model} &\textbf{Read Noise}\\ \hline
%              Camsci & Basler ACE acA640-750um CMOS  & 12.25 $e^-$ read noise\\ \hline
%              Camzyla & Zyla 4.2+ sCMOS detector & 7.98 $e^-$ read noise \\ \hline
%              Cam4p & Basler ace acA720-520um CMOS & 0.11 $e^-$ read noise \\ \hline
%              Camtip & Basler ACE acA640-750um CMOS (DOUBLE CHECK)  & 9.30 $e^-$ read noise\\ \hline
				
% 			\end{tabular}
% 		\end{center}
% 	\caption{CACTI components}
% 	\label{tab:CACTItable}
% \end{table}


\section{Experimental Details}


The CACTI testbed was used to compare the performance of a 3PWFS and 4PWFS in varying strengths of turbulence. Previous work by Schatz et al. found in simulation that the performance of the two wavefront sensors are comparable. These simulations however were run for only one seeing condition. The goal of this experiment is to determine the relative performance of the 3PWFS to the 4PWFS in varying strengths of turbulence using both the Raw Intensity and Slope Maps signal processing methods. The performance was determined by measuring the relative Strehl ratio of the AO corrected PSF and the aberration free PSF. For the CACTI testbed our aberration free PSF was the flat wavefront PSF on the science camera created by the deformable mirror shown in Figure \ref{fig:flatCACTI}.A., which we will refer to as the reference PSF. The reference PSF was used in the calibration of each PWFS. To calculate the Strehl Ratio we developed a Strehl Calculation pipeline detailed in Appendix A. The Strehl ratio is calculated by taking the ratio of the peak intensity of the partially corrected PSF, $P_{data}$,  to the peak intensity of the reference PSF, $P_0$. The peaks are normalized by the flux in the image. 

\begin{equation}
    S=\frac{P_{data}/Flux_{data}}{P_{0}/Flux_{0}}
    \label{Strehl}
\end{equation}

The first experiment performed on CACTI was to measure the Strehl Ratio as a function of turbulence strength and modulation radius for both the 3PWFS and the 4PWFS.  This experiment was performed using the Raw Intensity signal handling method. The modulation radii used were: 1.6 $\lambda/D$, 3.25 $\lambda/D$, and 5 $\lambda/D$. At each modulation radius the following experiment was performed. For all experiments the loop speed was 400-Hz.

\begin{itemize}
    \item Apply the best flat commands on the DM and create the reference PSF from the average of 1000 frames
    \item Dark frames taken for each of the cameras. Dark frames created from an average of 1000 frames
    \item A new calibration and reconstructor matrix was taken for each modulation radius.
    \item Apply simulated turbulence at levels 0.1-$\mu m$, 0.2-$\mu m$, 0.3-$\mu m$, 0.4-$\mu m$, 0.5-$\mu m$ RMS wavefront error. 
    \item At each level of turbulence strength, generate 5 different phase screens
    \item At each phase screen record 50 closed-loop PSF images, where each image is created from taking the average from 200 frames of data. 
    \item Calculate the average Strehl value from the closed-loop PSFs and average the Strehl ratio over the 5 trails to create an average Strehl value for each turbulence strength.
    \item Plot the Strehl value as a function of turbulence strength and magnitude radius. 
    

    
\end{itemize}





	
 


%An adaptive optics system works on the concept of conjugate imaging. An object plane corresponding to a layer of atmospheric turbulence is imaged through the telescope AO system onto a deformable mirror where the correction is applied. To achieve this the bulk of an AO system on sky or in the lab consists of pupil relays.

% \begin{figure}
%     \centering
%     \includegraphics[width=1\textwidth]{psfs.png}
%     \caption{PSFs on the science camera with A) Flat Wavefront, B) Turbulence, C) AO Corrected Turbulence.}
%     \label{fig:PSF}
% \end{figure}

 

% \begin{figure}
%     \centering
%     \includegraphics[width=0.5\textwidth]{3poptic.png}
%     \caption{The 3D model of the refractive 3 side pyramid optic.}
%     \label{fig:3poptic}
% \end{figure}




\section{Results}
For our setup, we found that 3PWFS performed better at higher modulations than the 4PWFS. At low modulation the performance of both PWFS was comparable for low turbulence strength, and at high turbulence the 4PWFS outperformed the 3PWFS. Figure \ref{fig:results} shows the resulting Strehl as a function of turbulence strength for modulations radii of 1.6 $\lambda/D$, 3.25 $\lambda/D$, and 5 $\lambda/D$. In these results, the Strehl ratio is the relative Strehl ratio in which the AO-corrected PSF is compared to the reference PSF. The data was processed by the Strehl calculation tool detailed in Appendix \ref{strehlct}. The difference in Strehl of the wavefront sensors varies on average by 0.02 Strehl at 1.6 $\lambda/D$, 0.02 Strehl at 3.25 $\lambda/D$, and 0.03 Strehl at 5 $\lambda/D$ modulation. It is difficult to perform a direct comparison of these PWFS because any misalignments, non-common path errors in the camera lenses, and other challenges result in a change in system performance. The result is that the Strehl ratio recorded is worse than the optimized performance. An effort was made to minimize these errors by taking all of the data on the same day, except for the 3PWFSRI 5 $\lambda/D$ modulation data which was taken two days after. From this data, we cannot conclude definitively that the 3PWFS performs better than the 4PWFS. The results show that the 3PWFS has comparable performance to the 4PWFS. 


\begin{figure}
    \centering
    \includegraphics[width=0.8\textwidth]{StrehlvsTurbRI4vs3.png}
    \caption{ Strehl as a function of turbulence strength for modulations radii of 1.6 $\lambda/D$, 3.25 $\lambda/D$, and 5 $\lambda/D$. In all trials the performance of both PWFS were comparable. A large gain in performance is seen when the modulation radius is reduced.}
    \label{fig:results}
\end{figure}

In our tests we found that the performance of both PWFS increased significantly at lower modulation. The gain in performance from the 3PWFS at 1.6 $\lambda/D$ was an average of 0.34 Strehl compared to 3.25 $\lambda/D$, and 0.28 Strehl compared to 5 $\lambda/D$. For the 4PWFS the gain in Strehl was and average of  0.38 Strehl compared to 3.25 $\lambda/D$, and 0.33 Strehl compared to 5 $\lambda/D$. The results demonstrate the loss in sensitivity due to higher modulation. 


\section{Discussion}

In the current configuration a 3PWFS and 4PWFS were integrated into CACTI for a performance test. Both PWFS were designed with refractive pyramid optics and had similar sampling across the pyramid pupils. Effort was put into minimizing the differences between each PWFS. Non-common path error was minimized by insuring each pyramid optic had the same PSF on the tip. The performance of each wavefront sensor was found to be and average of 0.02 Strehl at 1.6 $\lambda/D$, 0.02 Strehl at 3.25 $\lambda/D$, and 0.03 Strehl at 5 $\lambda/D$ modulation. The performance of the PWFS are too similar to decouple any differences from systematic errors such as misalignment or imperfect calibrations. Our results agree with Schatz et al, that the performance of the 3PWFS is comparable to the 4PWFS. 

The results of our experiment showed the gain in performance by decreasing the modulation radius. It has been shown that increasing the modulation radius decreases the sensitivity of the PWFS. (CITE SOMEONE IDK) Previous work on the theoretical sensitivity of the PWFS showed that the loss of sensitivity is maximum at the spatial frequencies at and below the spatial frequency of the modulation radius. (CITE GUYON, BOND). The result should be that the sensitivity of low order modes should be decreased, and that the sensitivity of high order modes is mostly preserved. The power of low order modes of the turbulence screens used by CACTI are filtered so that the full stroke of the DM is not used. Most of the power in the spatial frequencies of the turbulence screen power spectrum are mid to high spatial frequencies. Our results suggest that modulating decreases sensitivity across all spatial frequencies, and that the AO system should be run at as small of a modulation radius as possible. 



\section{Conclusion}

We have presented the design of the Comprehensive Adaptive Optics and Coronagraph Test Instrument (CACTI), a new ExAO testbed designed with the flexibility to support visiting instruments and to be easily re-configurable to perform multiple experiments. In the current configuration a visiting 3PWFS designed by Hartsci LLC was integrated into CACTI for a performance test Non-common path error was minimized by insuring each pyramid optic had the same PSF on the tip. We demonstrated the operation of the 3PWFS by closing the AO on simulated turbulence on CACTI. The performance of each wavefront sensor was determined by measuring the relative Strehl ratio of the closed loop PSF with the reference PSF. The Strehl ratio was calculated by a Strehl calculation tool developed for CACTI in Python. The difference in performance of each wavefront sensor was measured to be an average of 0.02 Strehl at 1.6 $\lambda/D$, 0.02 Strehl at 3.25 $\lambda/D$, and 0.03 Strehl at 5 $\lambda/D$ modulation. Our experiment showed the gain in performance by decreasing the modulation radius. Decreasing the modulation radius from 3.25 $\lambda/D$ to  1.6 $\lambda/D$ resulted in a gain in Strehl of 0.34 for the 3PWFS and 0.38 for the 4PWFS. From our results we conclude that the 3PWFS has comparable performance to the 4PWFS, and that an AO system should be run with as small of a modulation radius as possible to optimize performance. 






\appendix 
\section{Strehl Ratio Calculation} \label{strehlct}

An adaptive optics system compensates phase errors to return the resolution of the system to the diffraction limit of the telescope. The correction is not perfect, and we are interested in measuring the performance of the AO system. The Strehl ratio relates the residual uncorrected phase errors to the AO corrected PSF. The Strehl ratio is defined as, 

\begin{equation}
    S \cong e^{-\sigma^2}
\end{equation}

where $S$ is the Strehl ratio, and $\sigma^2$ is the residual RMS phase error. The measurement of the Strehl ratio for an AO corrected PSF is given in Equation \ref{Strehl}. The Strehl ratio is calculated by taking the ratio of the peak intensity of the partially corrected PSF, $P_{data}$,  to the peak intensity of an aberration free PSF, $P_0$. The peaks are normalized by the flux in the image. A simulated model can be used for the aberration free image. 

\begin{equation}
    S=\frac{P_{data}/Flux_{data}}{P_{0}/Flux_{0}}
    \label{Strehl}
\end{equation}

The calculation of Strehl for the data taken from the CACTI testbed was done using a Strehl ratio calculation tool that developed in Python. The flow of the Strehl ratio code is given in Figure \ref{fig:strehlClass}. The user inputs a list of filenames containing the image files (pathToData) and the image file for the aberration free PSF (pathToPerfect). For the CACTI testbed our aberration free PSF was the flat wavefront PSF on the science camera created by the deformable mirror, which we will refer to as the reference PSF. A simulated perfect PSF matching the plate scale of the system is an alternative. These two paths are the only mandatory inputs to the pipleline. If only one filename is given in pathToPerfect, then that reference PSF will be used in the Strehl calculation for every data file. If multiple files are used for the reference PSF, than the number of filenames for pathToPerfect must match that of pathToData. In this case, the reference PSF will be used with the data PSF from the filename with the same list index to calculate Strehl. An optional file to read in is pathToDarks which contains the dark files that will be used in dark subtraction. For CACTI the same camera is used to record the AO corrected PSFs and the reference PSF so the dark file is the same and used for both data sets. The pipleline checks that none of the data is saturated, by comparing the max of every image to the hard coded over-saturation value of the science camera which is a value 1023. If any of the data is over-saturated the pipeline with produce an error.

The Strehl ratio is calculated using the peak of the data. The Strehl calculation tool can find the peak of each image two different ways. The default method assumes that the highest value in each image is the peak of the data. An optional way is to use the Smart Peak Finder, which is an optional attribute provided by the user. The Smart Peak Finder uses the DAOStarFinder from Astropy that uses a Gaussian fitter to find the peak of the image. The user can provide the full-width half-max (FWHM) of the Gaussian kernel used by DAOStarFinder as an optional object attribute. If the user does not specify a FWHM, the Strehl Calculation tool will estimate the FWHM by pulling a sample image from the reference PSF data, and fitting a Gaussian to the a slice of the PSF created by taking the radial average around the max value in the frame. Once the peak value in each frame is found, the coordinates of that peak are saved for later use in the pipeline. 

The dark subtraction is performed in two steps. First if the user provided a dark files, they are subtracted off of frame. A second round of dark subtraction is done by masking out the PSF, and row by row taking the average value of the row, and subtracting it from that row. The mask is a circular mask centered on the coordinates of the peak found in the previous step. There is a default diameter to the mask which is one third the size of the smallest dimension of the images. A diameter can be provided as an optional input by the user. After dark subtraction the peak value is then pulled from every frame to be used in the Strehl calculation detailed in Equation \ref{Strehl}. The peak is normalized by the Flux in the image. To do this the Strehl Calculation tool sums the Flux in circles of increasing radii that are centered on the peak of the image. The result is a Strehl as a function of radius from the peak. Figure \ref{fig:StrehlVsRadiiExample} is a plot of the output from the Strehl Calculator tool, for a data set of closed loop PSFs from five different files, each file representing a single closed loop run. The turbulence strength for this data set was 0.3 microns RMS. As the radii of the summed Flux increases, the Strehl ratio flattens out which indicates a good background subtraction was done. If the background subtraction was poor, the Strehl ratio would increase as radius increased. Making the diameter of the PSF mask used for background subtraction smaller can help solve this issue. 

\begin{figure}
    \centering
    \includegraphics[width=0.8\textwidth]{StrehlVsRadiiExample.png}
    \caption{Out put of the Strehl calculation tool. For each filename given, the data in the file is used to calculate a mean Strehl value. The value is plotted again the radius of the circle used to sum the Flux for normalization. Having the Strehl value converge to a flat value as shown, is an indicator that the background subtraction was done correctly. }
    \label{fig:StrehlVsRadiiExample}
\end{figure}





\begin{figure}
    \centering
    \includegraphics[scale=0.5]{strehlclass2.png}
    \caption{Flow and decision tree of the Strehl calcualation tool. The Strehl can be calculated two ways. One way assumes that the max value in the image is the peak of the PSF. The second way is smarter, and uses DAOStarFinder tool from Astropy that uses a Gaussian fitter to find the peak of the image. The final product is Strehl as a function of the radius of the circle used to sum the Flux for normalization. }
    \label{fig:strehlClass}
\end{figure}

\newpage


%%%%% References %%%%%

\bibliography{report}   % bibliography data in report.bib
\bibliographystyle{spiejour}   % makes bibtex use spiejour.bst

%%%%% Biographies of authors %%%%%

% \vspace{2ex}\noindent\textbf{First Author} is an assistant professor at the University of Optical Engineering. He received his BS and MS degrees in physics from the University of Optics in 1985 and 1987, respectively, and his PhD degree in optics from the Institute of Technology in 1991.  He is the author of more than 50 journal papers and has written three book chapters. His current research interests include optical interconnects, holography, and optoelectronic systems. He is a member of SPIE.

% \vspace{1ex}
% \noindent Biographies and photographs of the other authors are not available.

\listoffigures
\listoftables

\end{spacing}
\end{document}
